\section{Related Works}
The quest to understand dark energy and its connection to the vacuum has a rich history, drawing upon established cosmological concepts and ongoing debates. This section explores some key areas of related work that inform our investigation.

Dark Energy: The central theme of this thesis revolves around dark energy. Since its discovery in the late 1990s through observations of distant supernovae, dark energy has captivated researchers. Its repulsive nature, causing the universe's expansion to accelerate, necessitates a new form of energy beyond the realm of traditional matter and radiation. Understanding its origin and properties remains a primary goal in cosmology.

Observable Universe:  Our investigation is limited by the observable universe, the region of space from which we can detect light and other forms of radiation.  The observable universe provides the data and constraints that guide our theories about dark energy and the wider cosmos.

Hubble's Law:  One of the cornerstones of modern cosmology, Hubble's Law establishes a relationship between the distance of a galaxy and its recessional velocity. Galaxies farther away are observed to be moving away from us at a faster rate, indicating the universe's expansion. This expansion serves as a key piece of evidence for the existence of dark energy.

Cosmological Constant Problem:  A significant challenge in linking vacuum energy to dark energy stems from the cosmological constant problem. Quantum field theory predicts a vacuum energy density many orders of magnitude greater than the observed dark energy density. Reconciling this vast discrepancy remains a hurdle in the vacuum energy-dark energy connection.

Zero-Point Energy:  Closely related to vacuum energy is the concept of zero-point energy. This refers to the minimum energy state of a quantum field, even in the absence of any particles.  The properties of zero-point energy are intertwined with those of vacuum energy, and understanding one sheds light on the other.

Friedmann Equations:  These equations, derived by Alexander Friedmann, describe the expansion of the universe over time. They incorporate the concept of dark energy as a component influencing the universe's expansion rate. Examining these equations through the lens of vacuum energy can provide further insights into the potential link.

By critically evaluating these related works, we can establish a strong foundation for investigating the potential link between vacuum energy and dark energy. It allows us to build upon existing knowledge, address known challenges, and explore promising avenues for further research.