\section{Vacuum Energy}

Zero-point energy (ZPE)

\[ ZPE = \frac{1}{2} \hbar \omega = \frac{1}{2} hv = \frac{1}{2} h \frac{c}{\lambda} \]

While these equivalences provide ways to calculate ZPE for specific systems, calculating the total vacuum energy density due to ZPE across the entire universe is a complex issue in cosmology. Theoretical predictions differ significantly from observed dark energy density, leading to the "cosmological constant problem."

From the following estimated data, we can derive the vacuum energy density of the observable universe.

\begin{itemize}
    \item Observable Universe Diameter (\(L_{U}\)): \( 8.794 \cdot 10^{26} \si{m} \)
    \item Observable Universe Volume (\(V_{U}\)): \( 3.5611 \cdot 10^{80} \si{m^{3}} \)
    \item Estimated Hubble Constant (H): \( 67.8 \si{km.s^{-1}.Mpc^{-1}} \approx 2.1972 \cdot 10^{-18} \si{s^{-1}} \)
    \item Estimated Vacuum Energy Density (\(P_{vac^{estimated}}\)): \( 3.35 \cdot 10^{9} \si{eV.m^{-3}} \)
\end{itemize}

The theoretical prediction for vacuum energy density, based on quantum field theory, vastly exceeds the observed density of dark energy, a phenomenon known as the cosmological constant problem.  While a definitive calculation of vacuum energy density within the current framework may be elusive, this thesis explores alternative approaches to investigate the potential connection between these concepts.  By analyzing existing data, considering alternative theoretical frameworks, and exploring related concepts, this thesis aims to contribute to our understanding of dark energy and its possible link to the energy inherent in the vacuum.

This thesis initially explores the concept of vacuum energy by assuming 
a minimum wavelength defined by the Planck length and a maximum wavelength corresponding to the observable universe.  
However, limitations exist in this approach, as the nature of the vacuum field might not be constrained by such specific boundaries.
Therefore, we will shift our focus to calculating the vacuum energy density within a defined volume.
By employing established methods (or proposing a new approach), we aim to investigate the potential contribution of vacuum energy to the observed dark energy.
\par
Assumptions:
\begin{itemize}
    \item The minimum wavelength of vacuum energy is Planck length (\(L_{p}\))
    \[\text{The maximum frequency } v_{max} = \frac{c}{L_{p}} \approx 1.854885 \cdot 10^{43} \]
    \item The maximum wavelength of vacuum energy is observable universe diameter (\(L_{U}\))
    \[\text{The minimum frequency } v_{min} = \frac{c}{L_{U}} \approx 3.4000 \cdot 10^{-19}\]
\end{itemize}


Therefore

\[P_{vac} = \frac{\sum\limits_{n}E_{n}}{V_{U}} = \frac{\sum\limits_{n}n\frac{hv}{2}}{V_{U}}\]
\begin{equation}
\begin{aligned}
P_{vac} = \frac{h}{2} \times \frac{(v_{max} + v_{min})\times(v_{max}-v_{min})}{v_{min} \times 2} / V_{U}\\
\approx \frac{h}{2} \times 5.059703 \cdot 10^{104} / V_{U}\\
\approx 1.0462626 \cdot 10^{90} \si{eV} / V_{U}\\
\approx 2.938032044 \cdot 10^{9} \si{eV.m^{-3}}\\
\end{aligned}
\end{equation}

Which is close to estimated vacuum energy density \(P_{vac^{estimated}}\).
