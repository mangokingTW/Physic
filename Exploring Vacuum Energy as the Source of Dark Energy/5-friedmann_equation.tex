\section{Friedmann equation}

The Friedmann equation is an equation derived from general relativity that describes the expansion of the universe over time. It relates the expansion rate (represented by Hubble's constant, H) to the properties of the universe, including its density and energy content.

\[ H^2 = \frac{8 \pi G}{3} \rho + \frac{\Lambda c^2}{3} \]

Cosmological Constant (Λ): Introduced by Einstein in his general theory of relativity, the cosmological constant is a constant term added to the Friedmann equations. It represents a mysterious form of energy that contributes to the expansion of the universe.

\[ \Lambda = \frac{8 \pi G}{c^2} P_{vac} \]

We can transform the Friedmann equation to the following form

\[ H^2 = \frac{8 \pi G}{3}(\rho + P_{vac}) \]

If vacuum energy is indeed the source of dark energy, then calculating its density becomes paramount. This thesis undertakes a meticulous exploration of this calculation, aiming to illuminate the potential link between these two enigmatic concepts.