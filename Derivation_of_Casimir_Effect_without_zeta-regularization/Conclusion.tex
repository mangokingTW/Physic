\section{CONCLUSION}


This derivation for the Casimir effect without zeta-regularization is that the average vacuum energy density 
(energy per unit area, \(\langle E \rangle/A\) ) between two perfectly conducting plates separated by distance a is:
\begin{align}\frac{\hbar c \pi^2}{720 a^3}\end{align}

This paper's derivation of the Casimir effect, lacking the negative sign,  
differs from the standard approach which attributes the negative sign to the existence of negative energy between the plates. 

While this paper calculates the absolute value of the energy density, the standard derivation focuses on the energy difference, 
where the negative sign reflects the lower energy state within the cavity due to the presence of the plates.

Since we previously established that the energy inside the plates is negative, 
the expected vacuum energy (represented by $\langle E \rangle$) is also negative.
\begin{align}
    \frac{\langle E \rangle}{A} = -\frac{\hbar c \pi^2}{720 a^3}
\end{align}

The current work has explored the theoretical implications of negative pressure within the Casimir effect. 
A crucial next step is to definitively address the question of whether negative energy truly exists.