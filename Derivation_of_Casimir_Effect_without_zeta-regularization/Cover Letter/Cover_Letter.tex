\documentclass[aip,amsmath,amssymb,reprint]{revtex4-2}
\usepackage{amsmath}
\usepackage{amsfonts}
\usepackage{url}
\usepackage{natbib}
\usepackage{xeCJK}
\setCJKmainfont{SimSun}
\setCJKmonofont{SimSun}
\setCJKsansfont{SimSun}
\begin{document}

\preprint{AIP/123-QED}
\title{Cover Letter:\\Derivation of Casimir Effect without Zeta-Regularization}

\date{\today}

\begin{abstract}
We are submitting our manuscript entitled "Derivation of Casimir Effect without Zeta-Regularization" for your consideration for publication in Applied Physics Letters (APL).

This paper presents a novel derivation of the Casimir effect that circumvents the need for zeta-regularization, a technique traditionally employed to handle infinities arising during the calculations. 

Our approach relates the expected vacuum energy density between two perfectly conducting plates to the changes in length and position associated with each energy state confined within the plates. The uncertainty principle is incorporated to calculate the area linked to each state.

The final result aligns with the standard Casimir effect formula obtained with zeta-regularization. This work demonstrates the possibility of deriving the Casimir effect without relying on this technique, offering an alternative perspective on this well-established phenomenon.

We believe this manuscript is well-suited for APL due to its focus on a fundamental quantum effect with potential implications for miniaturized devices and nanotechnology. The derivation presented is concise and avoids complex mathematical techniques, making it accessible to a broad readership within the applied physics community.

Thank you for your time and consideration. We look forward to hearing from you soon.
\\
\\
Sincerely,

\noindent Corresponding Author: Ching-Hsuan Yen (顏靖軒)

\noindent Affiliation: Cathay Development Center Kaohsiung, Cathay Financial Holdings, Kaohsiung 806, Taiwan

\noindent E-mail: yench@cs.nctu.edu.tw

\end{abstract}

\maketitle

\end{document}
