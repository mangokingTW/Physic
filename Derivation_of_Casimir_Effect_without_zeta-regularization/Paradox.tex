\section{PARADOX OF CASIMIR EFFECT}


\subsection*{Precondition}
\begin{enumerate}
    \item \(L_o\) - Distance outside the plates
    \item \(L_i\) - Distance inside the plates (gap between the plates)
    \item \(P_o\) - Pressure outside the plates
    \item \(P_i\) - Pressure inside the plates
\end{enumerate}

\subsection*{Derivation}
Consider the scenario where  $L_o$ approaches infinity, while $L_i$ remains very small.
\begin{align}
    L_o \rightarrow \infty \\
    L_o \gg L_i
\end{align}

Since $L_o$ becomes infinitely large, the pressure outside plates($P_o$) remain constant regardless of the size of $L_i$.

\noindent In simpler terms:
\begin{align}L_o - Li \rightarrow L_o\end{align}

According to the Casimir effect formula\citep*{Casimir:1948dh}, the pressure difference between inside and outside the plates is:
\begin{align}P_i - P_o = -\frac{\hbar c \pi^2}{240 L_i^4}\end{align}

This equation suggests that $P_o$ is greater than $P_i$

\subsection*{Situation 1}
\begin{align}
    &L_i \rightarrow 0 \\
    &P_i \ge 0 \\
    &P_o = P_i + \frac{\hbar c \pi^2}{240 L_i^4} \rightarrow {\mathbb{R}}^+ + \infty = \infty
\end{align}

This scenario leads to an unrealistic outcome. The pressure outside the plates ($P_o$) would tend towards infinity.
Consequently, the pressure difference predicted by the Casimir effect would become infinitely negative regardless of the gap size ($L_i$).
\begin{align}P_i - P_o \rightarrow P_i - \infty = - \infty\end{align}

This outcome contradicts our observations in the real world.

\subsection*{Situation 2}
\begin{align}
    &P_i < 0 \\
    &P_i \rightarrow -\frac{\hbar c \pi^2}{240 L_i^4} \\
    &P_o = P_i + \frac{\hbar c \pi^2}{240 L_i^4} \rightarrow -\frac{\hbar c \pi^2}{240 L_i^4} + \frac{\hbar c \pi^2}{240 L_i^4} = 0
\end{align}

This solution implies a negative pressure inside the platese ($P_i < 0$), 
which can be interpreted as \textbf{negative energy density} within that region.
Using the upper limit of the cosmological constant, 
the vacuum energy of free space (similar to $P_o$) has been estimated to be $5.26\times10^{-10}$(J/m$^3$)  \citep*{collaboration2020planck}.
While this solution aligns better with our understanding of the universe, 
the existence of negative energy itself remains an open question and requires further investigation.

