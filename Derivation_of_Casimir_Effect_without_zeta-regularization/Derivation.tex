\section{Derivation}

\noindent\textbf{Precondition}


\begin{enumerate}
    \item Uncertainty Principle\citep*{heisenberg1930physical}:
        \begin{align}
            &\Delta E \Delta t \ge \frac{\hbar}{2}& 
            &\Delta x \Delta p \ge \frac{\hbar}{2}&
        \end{align}
    \item The conclusion of Casimir effect assuming zeta-regularization\citep*{Casimir:1948dh}: 
    \begin{align}&\frac{\langle E \rangle}{A} = -\frac{\hbar c \pi^2}{720 a^3}\end{align}
    where \(A\) is the area of the metal plates and $a$ is the distance between the metal plates.
    \item $a$ - The distance between two uncharged conductive plates in a vacuum.
    \item The plates lie parallel to the \(xy\)-plane and is orthogonal to \(z\)-axis.
\end{enumerate}

\noindent\textbf{Calculate the Expected Vacuum Energy}


$\langle E \rangle$ represents the expected vacuum energy, 
which is the average $\bar E$ of individual energy states ($E_n$) for all possible states ($n$) in the system.
\begin{align}
    &\langle E \rangle = \sum_{n=1}^{\infty}E_n = \sum_{n=1}^{\infty}\bar{E}
\end{align}

We can express the average energy $\bar E$ as the energy uncertainty of 
a state $\Delta E_n$ divided by the ratio of its existence time $\Delta t_n$ 
to a characteristic timescale $t$.
This timescale $t$ represents the time it takes energy to travel across the gap $a$ between the plates.
\begin{align}
    &t = \frac{a}{c} \\
    &\bar{E} = \Delta E_n\frac{\Delta t_n}{t}
\end{align}

A key concept from the Heisenberg uncertainty principle states that 
the uncertainty energy of a state ($\Delta E_n$) and its corresponding uncertainty in time ($\Delta t_n$) are 
related by a constant factor ($\hbar/2$).
\begin{align}
    &\Delta E_n \Delta t_n = \frac{\hbar}{2}\\
    &\langle E \rangle = \sum_{n=1}^{\infty} \Delta E_n \frac{\Delta t_n}{t} = \sum_{n=1}^{\infty} \frac{\hbar c }{2 a}
\end{align}
This expresses the expected vacuum energy density by summing 
the product of energy uncertainty and its corresponding time uncertainty ratio for all states.

\noindent\textbf{Calculate the Area}


The total area $A$ is the sum of the areas $A_n$ for each individual state $n$.
The concept behind this summation is that the overall area accessible to 
a particle is determined by considering the allowed areas for each possible quantum state.
\begin{align}
    A = \sum_{n=1}^{\infty}A_n
\end{align}

    The $n$th area $A_n$ relates to the change in length and the change in position.
    \begin{align}
        A_n = L_n^2 = (\frac{a}{\Delta x_n^{x,y} \cdot n^z} \cdot a)^2
    \end{align}

    The wavenumber $k_n$ associated with the $n$th state. It relates to the momentum and wavelength of the particle in that state.
    \begin{align}
        k_n = \frac{n \pi}{a}& &p_n = \hbar k_n = \frac{n \pi \hbar}{a}&
    \end{align}

    This equation calculates the change in position along the x and y directions $\Delta x_n^{x,y}$ for the $n$th state. 
It uses the Heisenberg uncertainty principle, which states that the product of momentum uncertainty and position uncertainty 
has a lower bound.
    \begin{align}
        \Delta x_n^{x, y} = \frac{\hbar/2}{p_n} = \frac{\hbar/2}{n \pi \hbar/a} = \frac{a}{2 n \pi}
    \end{align}
$n^z$ represents the ratio of a single wave in $n$th state passing through the z-axis to the plates.
    \begin{align}
        n^z = \frac{a}{2 n \pi} \cdot \frac{2 \pi}{a} =\frac{1}{n}
    \end{align}
    
Therefore

    \begin{align}
        &A_n = (\frac{a}{\Delta x_n^{x,y} \cdot n^z} \cdot a)^2 = 4 n^4 \pi^2 a^2& \\
        &A = \sum_{n=1}^{\infty}4 n^4 \pi^2 a^2&
    \end{align}

This summation considers all possible quantum states the particle can occupy within the system 
defined by the distance between the plates.

\noindent\textbf{Calculate Vacuum Energy on Two Plates}


\begin{align}
    &\frac{\langle E \rangle}{A} = \sum_{n=1}^{\infty}\Delta E_n\frac{\Delta t_n}{t}\frac{1}{A_n} = \sum_{n=1}^{\infty} \frac{\hbar c }{2 a} \cdot \frac{1}{4 n^4 \pi^2 a^2} \nonumber \\
    &= \sum_{n=1}^{\infty} \frac{\hbar c}{8 n^4 \pi^2 a^3} = \frac{\hbar c}{8 \pi^2 a^3} \sum_{n=1}^{\infty} \frac{1}{n^4} \nonumber \\
    &= \frac{\hbar c}{8 \pi^2 a^3} \zeta(4) = \frac{\hbar c}{8 \pi^2 a^3} \cdot \frac{\pi^4}{90} = \frac{\hbar c \pi^2}{720 a^3}
\end{align}
