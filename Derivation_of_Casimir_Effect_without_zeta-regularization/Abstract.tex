The Casimir effect describes the attractive force arising due to quantum fluctuations of the vacuum electromagnetic field 
between closely spaced conducting plates. 

Traditionally, zeta-regularization is employed in calculations to address infinities that emerge during the derivation. 
This paper presents a novel derivation of the Casimir effect that circumvents the need for zeta-regularization.

We derive a formula for the average vacuum energy density between two perfectly conducting plates separated by a distance. 
Our approach relates the expected vacuum energy to the change in length and position associated with each energy state. 
The uncertainty principle is incorporated to calculate the area linked to each state. 

The final result aligns with the standard Casimir effect formula obtained with zeta-regularization. 
This work demonstrates that the Casimir effect can be derived without relying on zeta-regularization, 
offering an alternative perspective on this well-established phenomenon.