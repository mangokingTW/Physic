\section{Introduction}
Intriguing theoretical frameworks, 
such as the Heisenberg Uncertainty Principle and the Casimir Effect, 
suggest the possibility of negative energy density existing within the quantum realm. 
Building upon this foundation, we posit that quantum fluctuations might not only create particle-antiparticle pairs, 
but also entities with negative mass. 
This paper explores the hypothetical scenario of negative mass black holes (NMBHs) arising from quantum fluctuations of 
these negatively massive particles.

NMBHs would be expected to exhibit characteristics fundamentally distinct from their positive mass counterparts. 
Unlike conventional black holes, NMBHs would possess an anti-gravitational nature, 
repelling rather than attracting matter. Additionally, the evaporation process through Hawking radiation, 
which is a defining feature of positive mass black holes, is predicted to be significantly impeded for NMBHs. 
Furthermore, due to the conservation of momentum, NMBHs are likely to resist merging with each other, 
suggesting they would remain at a microscopic scale.

This paper delves into the theoretical implications of NMBHs, 
investigating their potential formation mechanisms, stability, 
and interaction with surrounding matter. 
By exploring the properties of these exotic objects, 
we aim to contribute to the ongoing exploration of the boundaries of physics at the quantum level.