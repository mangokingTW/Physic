\section{Derivation}
\noindent\textbf{Let}
\begin{enumerate}
    \item The distance between two uncharged conductive plates in a vacuum \[a\]
    \item The wavenumber orthogonal to the plates \[k_n = \frac{n\pi}{a}\]
    \item The plates lie parallel to the \(xy\)-plane and is orthogonal to \(z\)-axis
\end{enumerate}

\noindent\textbf{Calculate the Expected Vacuum Energy}
\begin{align*}
    &\langle E \rangle = \frac{1}{2}\sum_{n=1}^{\infty}E_n = \frac{1}{2}\sum_{n=1}^{\infty}\Delta E_n\frac{\Delta t_n}{t}& \\
    &\omega_n = c \cdot k_n = n \cdot \frac{c \pi}{a}&\\
    &\Delta E_n = \frac{1}{2} \hbar \omega_n = n \cdot \frac{ \hbar c \pi}{2a}& \\
    &\frac{\Delta t_n}{t} = \frac{\hbar/2}{\Delta E_n}\cdot\frac{1}{t} = \frac{a}{n c \pi}\cdot\frac{c}{a} = \frac{1}{n \pi}&\\
    &\sum_{n=1}^{\infty} \Delta E_n \frac{\Delta t_n}{t} = \sum_{n=1}^{\infty} \frac{n \hbar c \pi}{2 a} \cdot \frac{1}{n \pi} = \sum_{n=1}^{\infty} \frac{\hbar c }{2 a}&\\
\end{align*}
\begin{itemize}
    \item \textbf{Summation notation:} Define the expected vacuum energy (denoted by $\langle E \rangle$ ) 
    as the average of individual energy states (represented by $E_n$). 
    Since we're summing over infinitely many states, summation notation is used.
    \item \textbf{Relating energy and time:} The key idea is that the energy uncertainty ($\Delta E_n$) is related 
    to the time uncertainty ($\Delta t_n$) through a constant factor ($\hbar/2$).
    \item \textbf{Introducing wavenumber:} Define the wavenumber ($k_n$) for the $n$th state, 
    which relates to the wave's frequency ($\omega_n$) and the distance between the plates ($a$).
    \item \textbf{Calculating energy uncertainty:} Calculate the energy uncertainty for each state ($\Delta E_n$)
     using the wavenumber and the proportionality constant from the uncertainty principle.
    \item \textbf{Deriving time uncertainty ratio:} Calculate the ratio of the time uncertainty for each state ($\Delta t_n$)
     to the total time period ($t$). This ratio is inversely proportional to the energy uncertainty.
    \item \textbf{Summing the expected energy:} This expresses the expected vacuum energy density by summing 
    the product of energy uncertainty and its corresponding time uncertainty ratio for all states, 
    divided by the total area (which will be calculated later). 
    This summation represents the average energy density across all possible quantum states.
\end{itemize}
\noindent\textbf{Calculate the Area}
\par
\[A = \sum_{n=1}^{\infty}A_n\]
The \(n\)th area \(A_n\) relates to the change in length and the change in position.
    \begin{align*}
    &A_n = a_n^2 = (\frac{L_n}{\Delta x_n^{x,y}} \cdot a)^2&\\
    &L_n = n a&\\
    &p_n = \hbar k_n = \frac{n \pi \hbar}{a}&\\
    \end{align*}
    
    The length of $x, y$ axis waves locating on the $n$th area is
    \[\Delta x_n^{x, y} = \frac{\hbar/2}{p_n} = \frac{\hbar/2}{n \pi \hbar/a} = \frac{1}{2 n \pi a}\]
    
    Therefore

    \[A_n = (\frac{L_n}{\Delta x_n^{x,y}} \cdot a)^2 = 4 n^4 \pi^2 a^2\]
\begin{itemize}
    \item \textbf{Relating area and uncertainty:} The concept is that the area for a particular state is related to 
    the change in length ($L_n$) and the change in position ($\Delta x_n^{x,y}$) along the x and y axes.
    \item \textbf{Defining state variables:} Define the change in length for the $n$th state ($L_n$) 
    and the momentum ($p_n$) associated with that state.
    \item \textbf{Calculating x,y position uncertainty:} Due to the confinement in the z-axis, 
    this calculates the change in position along the x and y directions for the $n$th state ($\Delta x_n^{x,y}$). 
    This is derived using the uncertainty principle and the momentum along those directions.
    \item \textbf{Expressing Area:} The formula for the area ($A_n$) of the $n$th state relates the area to the change in length, 
    change in position along x and y, and the distance between the plates.
    \end{itemize}
\noindent\textbf{Calculate Vacuum Energy on Two Plates}
\begin{align*}
    &\frac{\langle E \rangle}{A} = 2 \cdot \frac{1}{2}\sum_{n=1}^{\infty}\Delta E_n\frac{\Delta t_n}{t}\frac{1}{A_n}& \\
    &= \sum_{n=1}^{\infty} \frac{\hbar c }{2 a} \cdot \frac{1}{4 n^4 \pi^2 a^2} = \sum_{n=1}^{\infty} \frac{\hbar c}{8 n^4 \pi^2 a^3} = \frac{\hbar c}{8 \pi^2 a^3} \sum_{n=1}^{\infty} \frac{1}{n^4}&\\
    &= \frac{\hbar c}{8 \pi^2 a^3} \zeta(4) = \frac{\hbar c}{8 \pi^2 a^3} \cdot \frac{\pi^4}{90} = \frac{\hbar c \pi^2}{720 a^3}&
\end{align*}