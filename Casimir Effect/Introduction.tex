\section{Introduction}

The Casimir effect, a cornerstone of quantum field theory, predicts an attractive force arising between closely 
spaced conducting plates due to quantum fluctuations of the vacuum electromagnetic field. 
This seemingly counterintuitive phenomenon has been experimentally verified and holds significant 
implications for miniaturized devices and nanotechnology.

Traditionally, the derivation of the Casimir effect relies on zeta-regularization, 
a mathematical technique used to handle infinities that can arise in quantum field theory calculations. 
While zeta-regularization is a powerful tool, it can introduce complexities and may obscure the underlying physical principles.

This paper presents an alternative approach to deriving the Casimir effect that avoids the use of zeta-regularization. 
Our derivation focuses on the average vacuum energy density between two perfectly conducting plates separated by a distance. 

We achieve this by: 
Relating the expected vacuum energy to the change in length and position associated with each energy state confined between the plates. 
Incorporating the uncertainty principle to calculate the area linked to each energy state.
Through this approach, we arrive at a formula for the average vacuum energy density that aligns with the standard Casimir effect formula obtained with zeta-regularization. This work demonstrates that the key physical principles behind the Casimir effect can be understood without resorting to zeta-regularization, offering a potentially more transparent perspective on this fascinating phenomenon.